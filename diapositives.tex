\documentclass[%handout,
compress]{beamer}
\usepackage{times}
\usepackage{etex}
\usepackage{ragged2e}  %to justify text
\usepackage{euscript,amsmath,amssymb,pifont}
\usepackage{graphicx}
\usepackage{algorithmic, algorithm}
\usepackage{leftindex}
\usepackage{libertine}
\usepackage[libertine,
slantedGreek,
nosymbolsc,
nonewtxmathopt,
subscriptcorrection]{newtxmath}
%-------------------theme---------------------------------------
\parskip    2pt 
\tolerance  2500
\usetheme{Luebeck}
%\setbeamertemplate{blocks}[rounded][shadow=false]
\setbeamertemplate{navigation symbols}{}
\setbeamertemplate{headline}{}
\useinnertheme{circles}
\usepackage{fontawesome}
\usepackage{pifont}
%-------------------- latexdraw and pstricks figures --------------
\definecolor{color2b}{rgb}{0.6,0.8,1.0}
\usepackage{epsfig}
\usepackage{mathtools}
\usepackage[usenames,dvipsnames]{pstricks}
\usepackage{pstricks-add}
\usepackage{pgfplots,tikz}

\pgfplotsset{compat=1.11} 

\usepackage{pst-grad}
\usepackage{pst-plot} 
\usepackage{pst-node} 
\usepackage{pst-eucl} 
\usepackage{pst-coil} 
\usepackage{newcent}
\usepackage{mathrsfs} 

\definecolor{dblue}{HTML}{0455BF}
\definecolor{dgreen}{HTML}{02724A}
\definecolor{dgreen2}{HTML}{025951}
\definecolor{dred}{HTML}{D90404}
\definecolor{dviolet}{HTML}{7F00FF}
\definecolor{labelkey}{HTML}{025951}
\definecolor{refkey}{HTML}{025951}
\definecolor{orng}{HTML}{D35400}
\definecolor{pblue}{rgb}{0.1176,0.5647,1}
\definecolor{pgreen}{rgb}{0.1961,0.8039,0.1961}
\definecolor{pred}{rgb}{1.0,0.2706,0.0}
\definecolor{fred}{rgb}{0.98,0.40,0.93}
\definecolor{pyellow}{rgb}{1.0,0.6471,0.0}
%------------------------------------------------
\definecolor{MaCouleur}{rgb}{0.90,1.0,1.0}
\definecolor{grayP}{rgb}{0.0,0.2,0.5}
\definecolor{mgreen}{HTML}{2651A6}
\definecolor{mblue}{HTML}{2651A6}
\definecolor{mred}{HTML}{D90404}
\definecolor{olu}{rgb}{0.5,0.0,0.5}
\definecolor{bleu}{rgb}{0.0,0.0,0.4}
\definecolor{bleu1}{rgb}{0.0,1.0,1.0}
\definecolor{bleu2}{rgb}{0.0,0.85,1.0}
\definecolor{bleu3}{rgb}{0.0,0.7,1.0}
\definecolor{bleu4}{rgb}{0.2,0.2,1.0}
\definecolor{bleu5}{rgb}{0.5,0.0,1.0}
\definecolor{red2}{rgb}{0.50,0.3,0.0}
\definecolor{red5}{rgb}{0.40,0.4,1.0}
\definecolor{red3}{rgb}{0.75,0.0,0.0}
\definecolor{red4}{rgb}{1.0,1.0,0.92}
\definecolor{ncstatew}{rgb}{0.2,0.2,0.2}
\definecolor{lightgreen}{rgb}{0.88,0.0,0.28}
\newrgbcolor{dgreen}{0.00 0.55 0.00}
\newrgbcolor{nido}{0.8 0.2 0.8}
%%%%%%%%%%%%%%%%%%%%%%%%%%%%%%%%%%%%%%%%%%%%%%%%%%%%%%%%%%%%%%%%%%
% delimiters
%%%%%%%%%%%%%%%%%%%%%%%%%%%%%%%%%%%%%%%%%%%%%%%%%%%%%%%%%%%%%%%%%%
\usepackage[extdef=true]{delimset}
\DeclareMathDelimiterSet{\scal}[2]{
\selectdelim[l]<{#1}
\mathpunct{}\selectdelim[p]|
{#2}\selectdelim[r]>}
\DeclareMathDelimiterSet{\EC}[2]{
\mathsf{E}\selectdelim[l]({#1}
\mathpunct{}\selectdelim[p]|
{#2}\selectdelim[r])}
\renewcommand{\pair}{\delimpair<{[.],}>}
\newcommand{\menge}[2]{\bigl\{{#1}\mid{#2}\bigr\}} 
\DeclareMathDelimiterSet{\Menge}[2]{\selectdelim[l]\{
{#1}\selectdelim[m]|{#2}\selectdelim[r]\}}
%blank norm, pairing, and scalar product
\newcommand*\Cdot{{\mkern 1.6mu\cdot\mkern 1.6mu}}
%%%%%%%%%%%%%%%%%%%%%%%%%%%%%%%%%%%%%%%%%%%%%%%%%%%%%%%%%%%%%%%%%%
% maths fonts
%%%%%%%%%%%%%%%%%%%%%%%%%%%%%%%%%%%%%%%%%%%%%%%%%%%%%%%%%%%%%%%%%%
\newcommand{\RR}{\mathbb{R}}
\newcommand{\NN}{\mathbb{N}}
\newcommand{\HH}{\mathcal{H}}
\newcommand{\XX}{\EuScript{X}}
\newcommand{\GD}{\mathcal{H}}
\newcommand{\GG}{\mathfrak{H}}
\newcommand{\AS}{\mathsf{A}}
\newcommand{\KKS}{\ensuremath{\boldsymbol{\mathsf K}}}
\newcommand{\HHS}{\ensuremath{\boldsymbol{\mathsf H}}}
\newcommand{\XXS}{\ensuremath{\boldsymbol{\mathsf X}}}
\newcommand{\YYS}{\ensuremath{\boldsymbol{\mathsf{Y}}}}
\newcommand{\GGS}{\ensuremath{{\boldsymbol{\mathsf G}}}}
\newcommand{\XS}{\mathsf{X}}
\newcommand{\YS}{\mathsf{Y}}
\newcommand{\HS}{\mathsf{H}}
\newcommand{\GS}{\mathsf{G}}
\newcommand{\ZS}{\mathsf{Z}}
\newcommand{\nS}{{\mathsf{n}}}
\newcommand{\nnn}{\mathsf{n}\in\mathbb{N}}
\newcommand{\iS}{\mathsf{i}}
\newcommand{\dS}{\mathsf{d}}
\newcommand{\jS}{\mathsf{j}}
\newcommand{\kS}{\mathsf{k}}
\newcommand{\lS}{\mathsf{l}}
\newcommand{\LS}{\mathsf{L}}
\newcommand{\LLS}{\boldsymbol{\mathsf{L}}}
\newcommand{\TS}{\mathsf{T}}
\newcommand{\BE}{\EuScript{B}}
\newcommand{\FE}{\EuScript{F}}
\newcommand{\kut}{\boldsymbol{\EuScript{K}}}
\newcommand{\sad}{\boldsymbol{\EuScript{S}}}
%%%%%%%%%%%%%%%%%%%%%%%%%%%%%%%%%%%%%%%%%%%%%%%%%%%%%%%%%%%%%%%%%%
% various sets
%%%%%%%%%%%%%%%%%%%%%%%%%%%%%%%%%%%%%%%%%%%%%%%%%%%%%%%%%%%%%%%%%%
\newcommand{\pinf}{{+}\infty}
\newcommand{\minf}{{-}\infty}
\newcommand{\zeroun}{\intv[o]{0}{1}}
\newcommand{\rzeroun}{\intv[l]{0}{1}}
\newcommand{\lzeroun}{\intv[r]{0}{1}}
\newcommand{\RXX}{\intv{\minf}{\pinf}}
\newcommand{\RX}{\intv[l]0{\minf}{\pinf}}
\newcommand{\RP}{\intv[r]0{0}{\pinf}}
\newcommand{\RPP}{\intv[o]0{0}{\pinf}}
\newcommand{\RPX}{\intv0{0}{\pinf}}
\newcommand{\RPPX}{\intv[l]0{0}{\pinf}}
\newcommand{\RM}{\intv[l]0{\minf}{0}}
\newcommand{\RMM}{\intv[o]0{\minf}{0}}
\newcommand{\RMX}{\intv0{\minf}{0}}
\newcommand{\emp}{\varnothing}
\newcommand{\forallmu}{\forall^{\mu}}
\newcommand{\WC}{\ensuremath{{\mathfrak W}}}
\newcommand{\SC}{\ensuremath{{\mathfrak S}}}
%%%%%%%%%%%%%%%%%%%%%%%%%%%%%%%%%%%%%%%%%%%%%%%%%%%%%%%%%%%%%%%%%%
% operators
%%%%%%%%%%%%%%%%%%%%%%%%%%%%%%%%%%%%%%%%%%%%%%%%%%%%%%%%%%%%%%%%%%
\newcommand{\Sum}{\displaystyle\sum}
\newcommand{\Int}{\displaystyle\int}
\newcommand{\Frac}[2]{\displaystyle{\dfrac{#1}{#2}}} 
\newcommand{\minimize}[2]{\underset{\substack{{#1}}}
{\operatorname{minimize}}\;\;#2}
\newcommand{\infconv}{\mathbin{\mbox{\small$\square$}}}
\newcommand{\einfconv}{\mathbin{\mbox{\footnotesize$\boxdot$}}}
\newcommand{\pushfwd}%
{\ensuremath{\mbox{\Large$\,\triangleright\,$}}}
\DeclareMathOperator*{\Argmind}{Argmin}
\DeclareMathOperator{\Argmin}{Argmin}
\DeclareMathOperator{\argmin}{argmin}
\newcommand{\Id}{\mathsf{Id}}
\newcommand{\ID}{\boldsymbol{\mathsf{Id}}}
\newcommand{\moyo}[2]{\leftindex[I]^{#2}{#1}}
\DeclareMathOperator{\card}{card}
%\DeclareMathOperator{\conv}{conv}
%\DeclareMathOperator{\cconv}{\overline{conv}}
%\DeclareMathOperator{\qq}{\ensuremath{\mathscr{Q}}}
\DeclareMathOperator{\cran}{\overline{ran}}
\DeclareMathOperator{\ran}{ran}
\DeclareMathOperator{\cdom}{\overline{dom}}
\DeclareMathOperator{\dom}{dom}
\DeclareMathOperator{\intdom}{int\,dom}
\DeclareMathOperator{\epi}{epi}
\DeclareMathOperator{\essinf}{ess\,inf}
\DeclareMathOperator{\esssup}{ess\,sup}
%\DeclareMathOperator{\cepi}{\overline{epi}}
\DeclareMathOperator{\Fix}{Fix}
\DeclareMathOperator{\gra}{gra}
\DeclareMathOperator{\zer}{zer}
\DeclareMathOperator{\inte}{int}
%\DeclareMathOperator{\reli}{ri}
\DeclareMathOperator{\sri}{sri}
\DeclareMathOperator{\reli}{ri}
\DeclareMathOperator{\sign}{sign}
\DeclareMathOperator{\rec}{rec}
\DeclareMathOperator{\prox}{prox}
\DeclareMathOperator{\proj}{proj}
\DeclareMathOperator{\spc}{\overline{span}}
\newcommand{\EE}{\mathsf{E}}
\newcommand{\PP}{\mathsf{P}}
\DeclareMathOperator{\PE}{\overset{\diamond}{\mathsf{E}}}
\newcommand{\proxc}[2]{{#1}\diamond{#2}}
\DeclareMathOperator{\dft}{DFT}
\DeclareMathOperator{\idft}{IDFT}
%-------------------MATH MISC-----------------------------------
\renewcommand{\le}{\leqslant}
\renewcommand{\ge}{\geqslant}
\renewcommand{\leq}{\leqslant}
\renewcommand{\geq}{\geqslant}
\newcommand{\exi}{\exists\,}
\newcommand{\weakly}{\rightharpoonup}
\newcommand{\mae}{\text{\normalfont$\mu$-a.e.}}
\newcommand{\Pas}{\text{\normalfont$\PP$-a.s.}}
%------------------------------------------------------------------
\usetikzlibrary{decorations}
\usetikzlibrary{backgrounds,arrows,positioning,decorations.%
markings}
\usetikzlibrary{shapes,snakes}
\tikzstyle{every picture}+=[remember picture]
%\everymath{\displaystyle}
%% - Tikz box --------------------
\tikzstyle{mybox} = [draw=lightgreen, fill=white, very thick,
rectangle, rounded corners, inner sep=5pt, inner ysep=5pt]
\tikzstyle{fancytitle} =[fill=lightgreen!30, text=black, ellipse]
%-------------------general setting-----------------------------
\renewcommand\mathfamilydefault{\rmdefault}
\setbeamercolor*{structure}{bg=white,fg=mblue}
\setbeamercolor*{palette primary}{use=structure,fg=white,bg=red3} 
\setbeamercolor*{palette secondary}{use=structure,fg=mblue,%
bg=white} 
\setbeamercolor*{palette tertiary}{use=structure,fg=white,%
bg=mblue} 
\setbeamercolor*{palette quaternary}{fg=white,bg=black}
\setbeamercolor{section in toc}{fg=black,bg=white}
\setbeamerfont{title}{family=\sffamily,size=\Large}
%-------------------alert blocks--------------------------------
\AtBeginEnvironment{alertblock}{%
\setbeamercolor{itemize item}{fg=mred}
%\setbeamertemplate{itemize items}{\faCheckCircle}
\setbeamertemplate{itemize items}{\ding{226}}
\setbeamerfont{block title alerted}{family=\normalfont}
\setbeamercolor{footnote}{fg=mred}
\setbeamercolor{footnote mark}{fg=mred}
\setbeamercolor{local structure}{fg=mred}
}
\setbeamercolor{block title alerted}{bg=mred,fg=white}
%-----------------proposition blocks------------------------------
\setbeamercolor{block title proposition}{bg=red3,fg=white}
\AtBeginEnvironment{propositionblock}{%
\setbeamercolor{itemize item}{fg=mgreen}
%\setbeamertemplate{itemize items}{\faCheckCircle}
\setbeamertemplate{itemize items}{\color{red3}\ding{226}}
\setbeamerfont{block title proposition}{family=\normalfont}
\setbeamercolor{footnote}{fg=mgreen}
\setbeamercolor{footnote mark}{fg=mgreen}
\setbeamercolor{local structure}{fg=mgreen}
}
% ------------------assumption blocks------------------------------
\setbeamercolor{block title assumption}{bg=red2,fg=white}
\AtBeginEnvironment{assumptionblock}{%
\setbeamercolor{itemize item}{fg=mgreen}
%\setbeamertemplate{itemize items}{\faCheckCircle}
\setbeamertemplate{itemize items}{\color{black}\ding{226}}
\setbeamerfont{block title assumption}{family=\normalfont}
\setbeamercolor{footnote}{fg=mgreen}
\setbeamercolor{footnote mark}{fg=mgreen}
\setbeamercolor{local structure}{fg=mgreen}
}
% ------------------example blocks------------------------------
\setbeamercolor{block title example}{bg=red2,fg=white}
\AtBeginEnvironment{block}{%
\setbeamercolor{itemize item}{fg=mgreen}
%\setbeamertemplate{itemize items}{\faCheckCircle}
\setbeamertemplate{itemize items}{\color{black}\ding{226}}
\setbeamerfont{block title example}{family=\normalfont}
\setbeamercolor{footnote}{fg=mgreen}
\setbeamercolor{footnote mark}{fg=mgreen}
\setbeamercolor{local structure}{fg=mgreen}
}
%-------------------usual blocks--------------------------------
\setbeamercolor{block title}{bg=red2,fg=white}
\setbeamerfont{block title}{family=\normalfont}
\setbeamercolor{block body}{fg=black,bg=red4}%
\setbeamercolor{block body alerted}{fg=black,bg=red4}%
\setbeamercolor{block body example}{fg=black,bg=red4}%
\setbeamercolor{block body proposition}{fg=black,bg=red4}%
\setbeamercolor{block body assumption}{fg=black,bg=red4}%
\setbeamercolor{frametitle}{bg=gray!16!white,fg=mblue}
\setbeamercolor*{titlelike}{parent=palette primary}
%-------------------
\setbeamercolor{frametitle}{fg=white,bg=red3}
%\setbeamercolor{structure}{fg=black,bg=white}
%\setbeamercolor{title}{bg=red3}
%\setbeamercolor{title}{fg=white}
%\setbeamercolor{block title}{bg=red3,fg=white}
%\setbeamercolor{block body}{bg=violet!15,fg=black}
%\beamertemplatesolidbackgroundcolor{MaCouleur}
%Vertical shading of the slides:
\setbeamercolor{normal text}{bg=}
\setbeamertemplate{background canvas}[vertical shading]%
[top=white!15,bottom=red3!01]
\beamertemplatenavigationsymbolsempty

%-----------------------------------------------------------
\newtheorem{proposition}[theorem]{Proposition}
\newtheorem{remark}[theorem]{Remark}
\newtheorem{assumption}[theorem]{Assumption}
%-----------------------------------------------------------
%\AtBeginSection{\frame{\tableofcontents[current]}}
%\setbeamercovered{transparent} %grise textes non encore affiches 
%------------------------------------------------------------------
%%%%%%%%%%%%%%%%%%%%%%%%%%%%%%%%%%%%%%%%%%%%%%%%%%%%%%%%%%%%%%%%%%
% authors' colors
%%%%%%%%%%%%%%%%%%%%%%%%%%%%%%%%%%%%%%%%%%%%%%%%%%%%%%%%%%%%%%%%%%
\newcommand{\tdred}[1]{{\color{dred}#1}}
\newcommand{\tpred}[1]{{\color{pred}#1}}
\newcommand{\tdblue}[1]{{\color{dblue}#1}}
\newcommand{\tpblue}[1]{{\color{pblue}#1}}
\newcommand{\tdgreen}[1]{{\color{dgreen}#1}}
\newcommand{\tpgreen}[1]{{\color{pgreen}#1}}
\newcommand{\tdviolet}[1]{{\color{dviolet}#1}}
\newcommand{\torng}[1]{{\color{orng}#1}}
\newcommand{\tpyell}[1]{{\color{pyellow}#1}}
%%%%%%%%%%%%%%%%%%%%%%%%%%%%%%%%%%%%%%%%%%%%%%%%%%%%%%%%%%%%%%%%%%
% operators with colors
%%%%%%%%%%%%%%%%%%%%%%%%%%%%%%%%%%%%%%%%%%%%%%%%%%%%%%%%%%%%%%%%%%
\newcommand{\LR}{\djr{L}}
\newcommand{\LRk}{\djr{L_k}}
\newcommand{\BB}{\djb{B}}
\newcommand{\BBk}{\djb{B_k}}
\newcommand{\gb}{\djb{g}}
\newcommand{\gbk}{\djb{g_k}}

\setbeamertemplate{caption}[numbered]


\title[Random proximal methods%
\hspace{2em}\insertframenumber/\inserttotalframenumber]%
{Randomly activated proximal methods for nonsmooth convex 
minimization}
\author[Javier I. Madariaga]
{Patrick L. Combettes \& {\underline{Javier I. Madariaga}}\\
{\footnotesize }} 

\institute{ 
Department of Mathematics\\
North Carolina State University\\
Raleigh, NC 27695, USA\\[-1mm]
}

\date{\scriptsize{
\textbf{TAGMaC Fall 2024}\\
November 02, 2024}}

\subject{
TAGMaC Fall 2024
}
%\vspace{2mm}}
\begin{document}
\small

\begin{frame}
\titlepage
\centering{
\includegraphics[height=0.6cm]{ncsu.eps}}
\end{frame}

\begin{frame}{Background and motivation}
\begin{itemize}
\item\justifying
Throughout, $\tpblue{\HS},\tpblue{\GS_{1}},\ldots,
\tpblue{\GS_{\mathsf{p}}}$ are separable real Hilbert spaces. 
\item\pause
$\upGamma_0(\tpblue{\HS})$ denotes the class 
of lower semicontinuous convex functions 
$\tdgreen{\mathsf{f}}\colon\tpblue{\HS}\to\RX$ such that
\begin{equation}
\nonumber 
\dom\tdgreen{\mathsf{f}}=\menge{\mathsf{x}\in\tpblue{\HS}}
{\tdgreen{\mathsf{f}}(\mathsf{x})<\pinf}
\neq\emp.
\end{equation}
\item\pause
Let $\tdgreen{\mathsf{f}}\in\upGamma_0(\tpblue{\HS})$. The 
subdifferential of $\tdgreen{\mathsf{f}}$ is the operator
\begin{equation}
\nonumber
\tdgreen{\partial\mathsf{f}}\colon\tpblue{\HS}\to 2^{\tpblue{\HS}}
\colon\mathsf{x}\mapsto
\menge{\mathsf{x}^*\in\tpblue{\HS}}
{(\forall\mathsf{z}\in\tpblue{\HS})\;
\scal{\mathsf{z}-\mathsf{x}}{\mathsf{x}^*}+\tdgreen{\mathsf{f}}
(\mathsf{x})
\leq\tdgreen{\mathsf{f}}(\mathsf{z})} 
\end{equation}
and the proximity operator of $\tdgreen{\mathsf{f}}$ is 
\begin{equation}
\nonumber
\tdgreen{\prox_{\mathsf{f}}}\colon\tpblue{\HS}\to\tpblue{\HS}
\colon\mathsf{x}\mapsto\underset{\mathsf{z}\in\tpblue{\HS}}
{\text{argmin}}\;\biggl(\tdgreen{\mathsf{f}}(\mathsf{z})+
\dfrac{1}{2}\|\mathsf{x}-\mathsf{z}\|^2\biggr).
\end{equation}
\end{itemize}
\end{frame}

\begin{frame}{Background and motivation}
\begin{itemize}
\justifying
\item
Let $\tpblue{\mathsf{C}}$ be a nonempty closed convex subset of 
$\tpblue{\HS}$. Then $\tdgreen{\iota_{\mathsf{C}}}$ denotes the 
indicator function of $\tpblue{\mathsf{C}}$ and
$\tdgreen{\proj_{\mathsf{C}}}
=\tdgreen{\prox_{\iota_{\mathsf{C}}}}$
the projection operator onto $\tpblue{\mathsf{C}}$.
\item\pause
The underlying probability space is $(\upOmega,\FE,\PP)$.% and 
$\tpblue{\BE_{\HS}}$ denotes the Borel $\upsigma$-algebra of
$\tpblue{\HS}$.
An $\tpblue{\HS}$-valued random variable is a measurable 
mapping 
$x\colon(\upOmega,\FE)\to(\tpblue{\HS},\tpblue{\BE_{\HS}})$. 
%The $\upsigma$-algebra generated by a family $\upPhi$ of 
%random variables is denoted by $\upsigma(\upPhi)$
\item\pause
We use sans-serif letters to denote deterministic variables and
italicized serif letters to denote random variables. 
\end{itemize}
\end{frame}

\begin{frame}{Background and motivation}
\begin{exampleblock}{Problem~1}
\justifying
$\tpblue{\HS}$ is a separable real Hilbert space and 
$\tdgreen{\mathsf{f}}\in\upGamma_0(\tpblue{\HS})$. For every
$\kS\in\{1,\ldots,\mathsf{p}\}$, $\tpblue{\GS_{\kS}}$ is a 
separable real Hilbert space, 
$\tdgreen{\mathsf{g}_{\kS}}\in\upGamma_0(\tpblue{\GS_{\kS}})$, and
$0\neq\tdred{\LS_{\kS}}\colon\tpblue{\HS}\to\tpblue{\GS_{\kS}}$ is 
linear and bounded. It is assumed that
\begin{equation}
\nonumber 
\zer\brk3{\tdgreen{\partial\mathsf{f}}+\sum_{\kS=1}^\mathsf{p}
\tdred{\LS^*_{\kS}}\circ\tdgreen{\partial\mathsf{g}_{\kS}}\circ
\tdred{\LS_{\kS}}}\neq\emp.
\end{equation}
The task is to
\begin{equation}
\nonumber
\minimize{\mathsf{x}\in\tpblue{\HS}}{\tdgreen{\mathsf{f}}
(\mathsf{x})+\sum_{\kS=1}^\mathsf{p}\tdgreen{\mathsf{g}_{\kS}}
(\tdred{\LS_{\kS}}\mathsf{x})}
\end{equation}
and the set of solutions is denoted by $\ZS$.
\end{exampleblock}
\justifying\pause
It covers a wide range of minimization models in data analysis and
it is an essential tool in signal processing, statistics, inverse
problems, and machine learning.
\end{frame}

\begin{frame}
\frametitle{Background and motivation}
\justifying
We aim at designing {\bf proximal algorithms} which are 
{\bf stochastic} in the sense that they activate a randomly 
selected block of functions at each iteration\pause
\vspace{5mm}

Furthermore, they should satisfy: 
\begin{itemize}
\justifying 
\item[\bfseries R1:]
They {\bf guarantee the convergence} of the sequence of iterates to
a solution to Problem~1 without any additional
assumptions.
\item[\bfseries R2:] \pause
At each iteration, more than one
randomly selected function 
$(\tpblue{\mathsf{f}},\tpblue{\mathsf{g}_1},\ldots,
\tpblue{\mathsf{g}_{\mathsf{p}}})$ can be
activated. 
\item[\bfseries R3:] \pause
Knowledge of bounds on the norms of the linear operators
$(\tdred{\LS_{\kS}})_{1\leq\kS\leq\mathsf{p}}$ is not
required. 
\end{itemize} 
\end{frame}

\begin{frame}
\frametitle{Background and motivation}
\justifying
There does not seem to exist methods that satisfy simultaneously
\tdblue{\bfseries R1--R3}. 
\vspace{5mm}\pause

Our main contribution is to proposed 
three algorithmic frameworks that comply with 
\tdblue{\bfseries R1--R3}.
\vspace{5mm}\pause

The strategy consists in embedding Problem~1 into a
{\bf multivariate problem}.
\end{frame}

%%%%%%%%%%%%%%%%%%%%% GENERAL FRAMEWORK %%%%%%%%%%%%%%%%%%%%%%%%%%
\begin{frame}{General Framework}
\begin{block}{Problem~2}
\label{prob:2}
\begin{itemize}
\justifying
\item
Let $(\tpblue{\XS_{\iS}})_{1\leq\iS\leq\mathsf{m}}$ and
$(\tpblue{\YS_{\jS}})_{1\leq\jS\leq\mathsf{r}}$ be families of 
separable real Hilbert spaces with direct Hilbert sums 
$\tpblue{\XXS}
=\tpblue{\XS_1}\oplus\cdots\oplus\tpblue{\XS_{\mathsf{m}}}$ and
$\tpblue{\YYS}
=\tpblue{\YS_1}\oplus\cdots\oplus\tpblue{\YS_{\mathsf{r}}}$. 
\pause
\item
For every $\iS\in\{1,\ldots,\mathsf{m}\}$, let 
$\tdgreen{\mathsf{f}_{\iS}}\in\Gamma_0(\tpblue{\XS_{\iS}})$
and, for every $\jS\in\{1,\ldots,\mathsf{r}\}$, let 
$\tdgreen{\mathsf{h}_{\jS}}\in\Gamma_0(\tpblue{\YS_{\jS}})$, 
and let $\tdred{\mathsf{M}_{\jS\iS}}\colon\tpblue{\XS_{\iS}}
\to\tpblue{\YS_{\jS}}$ 
be linear and bounded. 
%\pause
%\item
%It is assumed that there exists 
%$\boldsymbol{\mathsf{u}}\in\tpblue{\XXS}$ such that
%$(\forall\iS\in\{1,\ldots,\mathsf{m}\})\;\mathsf{0}\in
%\tdgreen{\partial\mathsf{f}_{\iS}}(\mathsf{u}_{\iS})
%+\sum_{\kS=1}^{\mathsf{r}}
%\tdred{\mathsf{M}_{\kS\iS}^*}\bigl(
%\tdgreen{\partial\mathsf{h}_{\kS}}\bigl(\sum_{\jS=1}^{\mathsf{m}}
%\tdred{\mathsf{M}_{\kS\jS}}\mathsf{u}_{\jS}\bigr)\bigr)$.\pause 
\end{itemize}
\pause
The task is to 
\begin{equation}
\nonumber
\minimize{\boldsymbol{\mathsf{x}}\in\tpblue{\XXS}}
{\sum_{\iS=1}^\mathsf{m}\tdgreen{\mathsf{f}_{\iS}}
(\mathsf{x}_{\iS})+\sum_{\jS=1}^\mathsf{r}
\tdgreen{\mathsf{h}_{\jS}}
\biggl(\sum_{\iS=1}^\mathsf{m}
\tdred{\mathsf{M}_{\jS\iS}}\mathsf{x}_{\iS}\biggr)}.
\end{equation}
\end{block}
\end{frame}

\begin{frame}{General Framework}
The set of solutions to Problem~2 is denoted by
$\boldsymbol{\mathsf{Z}}$. \pause
Further, the projection operator onto the subspace 
\begin{equation}
\nonumber
\boldsymbol{\mathsf{V}}=
\Menge3{(\boldsymbol{\mathsf{x}},\boldsymbol{\mathsf{y}})
\in\tpblue{\XXS}\oplus\tpblue{\YYS}}
{(\forall\jS\in\{1,\ldots,\mathsf{r}\})\;\mathsf{y}_{\jS}=
\sum_{\iS=1}^\mathsf{m}\tdred{\mathsf{M}_{\jS\iS}}\mathsf{x}_{\iS}}
\end{equation}
is decomposed as $\proj_{\boldsymbol{\mathsf{\mathsf{V}}}}
\colon(\boldsymbol{\mathsf{x}},\boldsymbol{\mathsf{y}})\mapsto
(\tdred{\mathsf{Q}_{\mathsf{l}}}
(\boldsymbol{\mathsf{x}},\boldsymbol{\mathsf{y}}))_{1\leq
\mathsf{l}\leq\mathsf{m}+\mathsf{r}}$, 
where for every $\iS\in\{1,\ldots,\mathsf{m}\}$, 
$\tdred{\mathsf{Q}_{\iS}}\colon\tpblue{\XXS}\oplus\tpblue{\YYS}
\to\tpblue{\XS_{\iS}}$ and,
for every $\jS\in\{1,\ldots,\mathsf{r}\}$, 
$\tdred{\mathsf{Q}_{\mathsf{m}+\jS}}
\colon\tpblue{\XXS}\oplus\tpblue{\YYS}\to\tpblue{\YS_{\jS}}$.
\end{frame}

\begin{frame}{General Framework}
\begin{block}{Theorem~3...}
Douglas–Rachford method yields the following result.
\begin{itemize}
\item \pause
Let $\upgamma\in\RPP$, 
\item \pause
let $(\uplambda_{\nS})_{\nnn}$ be a sequence in $\left]0,2\right[$ 
such that $\inf_{\nnn}\uplambda_{\nS}>0$ and 
$\sup_{\nnn}\uplambda_{\nS}<2$,
\pause
\item 
let $\boldsymbol{x}_0$ and $\boldsymbol{z}_0$
be $\tpblue{\XXS}$-valued random variables,\pause
\item
let $\boldsymbol{y}_0$ and $\boldsymbol{w}_0$
be $\tpblue{\YYS}$-valued random variables, \pause 
\item 
set $\mathsf{D}=\{0,1\}^{\mathsf{m}+\mathsf{r}}\smallsetminus
\{\boldsymbol{\mathsf{0}}\}$, and let 
$(\boldsymbol{\varepsilon}_{\nS})_{\nnn}$ be 
identically distributed $\mathsf{D}$-valued random variables. 
\red{(**)}
\end{itemize}
\end{block}
\end{frame}

\begin{frame}{General Framework}
\begin{block}{...Theorem~3}
Iterate
\begin{flalign*}
&\begin{array}{l}
\text{for}\;\nS=0,1,\ldots\\
\left\lfloor
\begin{array}{l}
\text{for}\;\iS=1,\ldots,\mathsf{m}\\
\left\lfloor
\begin{array}{l}
x_{\iS,\nS+1}=x_{\iS,\nS}+\varepsilon_{\iS,\nS}
\bigl(\tdred{\mathsf{Q}_{\iS}}
(\boldsymbol{z}_{\nS},\boldsymbol{w}_{\nS})
-x_{\iS,\nS}\bigr)\\[1mm]
z_{\iS,\nS+1}=z_{\iS,\nS}+\varepsilon_{\iS,\nS}\uplambda_{\nS}
\bigl(\tdgreen{\prox_{\upgamma\mathsf{f}_{\iS}}}
(2x_{\iS,\nS+1}-z_{\iS,\nS})-x_{\iS,\nS+1}\bigr)
\end{array}
\right.\\
\text{for}\;\jS=1,\ldots,\mathsf{r}\\
\left\lfloor
\begin{array}{l}
y_{\jS,\nS+1}=y_{\jS,\nS}+\varepsilon_{\mathsf{m}+\jS,\nS}
\big(\tdred{\mathsf{Q}_{\mathsf{m}+\jS}}
(\boldsymbol{z}_{\nS},\boldsymbol{w}_{\nS})
-y_{\jS,\nS}\big)\\[1mm]
w_{\jS,\nS+1}=w_{\jS,\nS}\,+\varepsilon_{\mathsf{m}+\jS,\nS}
\uplambda_{\nS}
\bigl(\tdgreen{\prox_{\upgamma\mathsf{h}_{\jS}}}
(2y_{\jS,\nS+1}-w_{\jS,\nS})-y_{\jS,\nS+1}\bigr).
\end{array}
\right.
\end{array}
\right.\\
\end{array}
\nonumber
\end{flalign*}
\pause
Then $(\boldsymbol{x}_{\nS})_{\nnn}$ converges weakly $\Pas$ to a
$\boldsymbol{\mathsf{Z}}$-valued random variable.
\end{block}
\end{frame}

%%%%%%%%%%%%%%%%%%%% FRAMEWORK 1 %%%%%%%%%%%%%%%%%%%%%%%%%%%%%%%%%
\begin{frame}{Framework~1}
How can we use this general framework to solve Problem~1?\pause
\begin{equation}
\tag{GF}
\minimize{\boldsymbol{\mathsf{x}}\in\tpblue{\XXS}}
{\sum_{\iS=1}^\mathsf{m}\tdgreen{\mathsf{f}_{\iS}}
(\mathsf{x}_{\iS})+
\sum_{\jS=1}^\mathsf{r}\tdgreen{\mathsf{h}_{\jS}}
\biggl(\sum_{\iS=1}^\mathsf{m}
\tdred{\mathsf{M}_{\jS\iS}}\mathsf{x}_{\iS}\biggr)}.
\end{equation}
\begin{equation}
\tag{P1}
\minimize{\mathsf{x}\in\tpblue{\HS}}{\tdgreen{\mathsf{f}}
(\mathsf{x})+
\sum_{\kS=1}^\mathsf{p}\tdgreen{\mathsf{g}_{\kS}}
(\tdred{\mathsf{L}_{\kS}}\mathsf{x})}.
\end{equation}\pause
The same for $\mathsf{m} = 1$, 
$\mathsf{r} = \mathsf{p}$, 
$\tpblue{\XS_1}=\tpblue{\HS}$, $\tdgreen{\mathsf{f}_1}
=\tdgreen{\mathsf{f}}$, and 
$(\forall \kS\in\{1,\dots,\mathsf{p}\})$ $\tpblue{\YS_{\kS}} 
=\tpblue{\GS_{\kS}}$, 
$\tdred{\mathsf{M}_{\kS,1}}=\tdred{\mathsf{L}_{\kS}}$, and 
$\tdgreen{\mathsf{h}_{\kS}}=\tdgreen{\mathsf{g}_{\kS}}$.
\end{frame}

\begin{frame}{Framework~1}
\begin{block}{Proposition~4}
\label{p:3}
Set $\mathsf{D}=\{0,1\}^{1+\mathsf{p}}\smallsetminus
\{\boldsymbol{\mathsf{0}}\}$ and consider Theorem~3 with\pause
\begin{equation}
\nonumber
\begin{array}{l}
\text{for}\;\nS=0,1,\ldots\\
\left\lfloor
\begin{array}{l}
\tdviolet{q_{\mathsf{n}}}
=(\Id+\sum_{\kS=1}^{\mathsf{p}}\tdred{\LS_{\kS}^*}\circ
\tdred{\LS_{\kS}})^{-1}
\bigl(z_{\nS}+\sum_{\kS=1}^{\mathsf{p}}
\tdred{\LS_{\kS}^*}w_{\kS,\nS}\bigr)\\
\tdred{\mathsf{Q}_{1}}(\boldsymbol{z}_{\nS},\boldsymbol{w}_{\nS})
=\tdviolet{q_{\nS}}\\\pause
\text{for}\;\kS=1,\dots,\mathsf{p}\\
\left\lfloor
\begin{array}{l}
\tdred{\mathsf{Q}_{1+\kS}}
(\boldsymbol{z}_{\nS},\boldsymbol{w}_{\nS})
=\tdred{\LS_{\kS}}\tdviolet{q_{\nS}}.
\end{array}
\right.\\
\end{array}
\right.\\
\end{array}
\end{equation}
Then $(x_{\nS})_{\nnn}$ converges weakly $\Pas$ to a
$\ZS$-valued random variable. 
\end{block}
\end{frame}


%%%%%%%%%%%%%%%%%%%% FRAMEWORK 2 %%%%%%%%%%%%%%%%%%%%%%%%%%%%%%%%%
\begin{frame}{Framework 2}
Can we use (GF) to solve the original problem differently?
\pause
Set $\tpblue{\GGS}=\tpblue{\GS_1}\oplus\cdots\oplus
\tpblue{\GS_{\mathsf{p}}}$ and
\begin{equation}
\nonumber
\boldsymbol{\mathsf{W}}=
\Menge3{\boldsymbol{\mathsf{x}}\in\tpblue{\HS}\oplus\tpblue{\GGS}}
{(\forall\kS\in\{1,\ldots,\mathsf{p}\})\;\mathsf{x}_{\kS+1}=
\tdred{\mathsf{L}_{\kS}}\mathsf{x}_{1}}.
\end{equation}\pause
\begin{block}{Problem~5}
Set $\tdgreen{\mathsf{f}_{1}}=\tdgreen{\mathsf{f}}$ and, for every 
$\iS\in\{2,\ldots,\mathsf{p}+1\}$,
$\tdgreen{\mathsf{f}_{\iS}}=\tdgreen{\mathsf{g}_{\iS-1}}$. Denote 
by $\boldsymbol{\mathsf{x}}=(\mathsf{x}_{1},\dots,
\mathsf{x}_{\mathsf{p}+1})$ a generic element 
in $\tpblue{\HS}\oplus\tpblue{\GGS}$. The task is to
\begin{equation}
\nonumber
\minimize{
\boldsymbol{\mathsf{x}}\in\tpblue{\HS}\oplus\tpblue{\GGS}}
{\sum_{\mathsf{i}=1}^{\mathsf{p}+1} 
\tdgreen{\mathsf{f}_{\mathsf{i}}}
(\mathsf{x}_{\mathsf{i}})\pause+
\tpyell{\iota_{\boldsymbol{\mathsf{W}}}}(\boldsymbol{\mathsf{x}})}.
\end{equation} 
\end{block}
\end{frame}

\begin{frame}{Framework 2}
\begin{exampleblock}{Proposition~6}
\label{p:1}
Set $\mathsf{D}=\{0,1\}^{\mathsf{p}+2}\smallsetminus
\{\boldsymbol{\mathsf{0}}\}$ and consider Theorem~3 with\pause
\begin{flalign*}
&\begin{array}{l}
\text{for}\;\nS=0,1,\ldots\\
\left\lfloor
\begin{array}{l}
\text{for}\;\iS=1,\ldots,\mathsf{p}+1\\
\left\lfloor
\begin{array}{l}
\tdred{\mathsf{Q}_{\iS}}
(\boldsymbol{z}_{\nS},\boldsymbol{w}_{\nS})
= \frac{1}{2}(z_{\iS,\nS}+w_{\iS,\nS})
\end{array}
\right.\\[0.7mm]
\only<3->{\tdred{\mathsf{Q}_{\mathsf{p}+2}}
(\boldsymbol{z}_{\nS},\boldsymbol{w}_{\nS}) 
=\frac{1}{2}(\boldsymbol{z}_{\nS}+\boldsymbol{w}_{\nS})}\\[0.7mm]
\only<1,2>{\phantom{\tdviolet{q_{\mathsf{n}}}
=(\Id+\sum_{\kS=1}^{\mathsf{p}}\tdred{\LS_{\kS}^*}\circ
\tdred{\LS_{\kS}})^{-1}
\bigl(2y_{1,\nS+1}-w_{1,\nS}
+\sum_{\kS=1}^{\mathsf{p}}\tdred{\LS_{\kS}^*}
(2y_{\kS+1,\nS+1}-w_{\kS+1,\nS})\bigr)}}
\only<3->{\tdviolet{q_{\mathsf{n}}}
=(\Id+\sum_{\kS=1}^{\mathsf{p}}\tdred{\LS_{\kS}^*}\circ
\tdred{\LS_{\kS}})^{-1}
\bigl(2y_{1,\nS+1}-w_{1,\nS}
+\sum_{\kS=1}^{\mathsf{p}}\tdred{\LS_{\kS}^*}
(2y_{\kS+1,\nS+1}-w_{\kS+1,\nS})\bigr)}\\[0.7mm]
\only<3->{\tpyell{\prox_{\iota_{\boldsymbol{\mathsf{W}}}}}
(2y_{\kS,\nS+1}-w_{\kS,\nS}) 
= (\tdviolet{q_{\nS}},\tdred{\LS_{1}}\tdviolet{q_{\nS}},\dots,
\tdred{\LS_{\mathsf{p}}}\tdviolet{q_{\nS}})}
\end{array}
\right.\\
\end{array}
\nonumber
\end{flalign*}
\pause\pause
Then $({x}_{1,\nS})_{\nnn}$ converges weakly $\Pas$ to a 
$\ZS$-valued random variable. 
\end{exampleblock}
\end{frame}

\section{F3}
\begin{frame}{Framework~3}
We extend the same idea...
\begin{block}{Problem~7}
\label{prob:6}
Consider the problem
\begin{equation}
\nonumber
\minimize{\boldsymbol{\mathsf{x}}\in\tpblue{\HS}\oplus
\tpblue{\boldsymbol{\mathsf{G}}}}
{\sum_{\mathsf{i}=1}^{\mathsf{p}+1}
\tdgreen{\mathsf{f}_{\mathsf{i}}}(\mathsf{x}_{\mathsf{i}})
\pause
+\sum_{\mathsf{k}=1}^{\mathsf{p}}\tpyell{\iota_{\{\mathsf{0}\}}}
\Biggl(\sum_{\iS=1}^{\mathsf{p}+1}\tpyell{\mathsf{C}_{\kS\iS}}
{\mathsf{x}_{\iS}}\Biggr)},
\end{equation}
where
\begin{equation}
\nonumber
\tpyell{\mathsf{C}_{\kS\iS}}=
\begin{cases}
\phantom{-}\tdred{\mathsf{L}_{\kS}}, & \text{if}\;\iS=1;\\
-\Id,&\text{if}\;\iS=\kS+1;\\
\phantom{-}\mathsf{0},& \text{otherwise.}
\end{cases}
\end{equation}
\end{block}
\end{frame}

%%%%%%%%%%%%%%%%%%%% FRAMEWORK 3 %%%%%%%%%%%%%%%%%%%%%%%%%%%%%%%%%
\begin{frame}{Framework~3}
\begin{exampleblock}{Proposition~8}
\label{p:11}
Set $\mathsf{D}=\{0,1\}^{2\mathsf{p}+1}\smallsetminus
\{\boldsymbol{\mathsf{0}}\}$ and consider Theorem~3 with\pause
\begin{flalign*}
&\begin{array}{l}
\text{for}\;\nS=0,1,\ldots\\
\left\lfloor
\begin{array}{l}
\tdviolet{{q}_{\nS}}=(2\Id+\sum_{\kS=1}^{\mathsf{p}}
\tdred{\LS_{\kS}^*}\circ\tdred{\LS_{\kS}})^{-1}
\bigl(2\mathsf{z}_{1,\nS}+
\sum_{\kS=1}^{\mathsf{p}}\tdred{\LS_{\kS}^*}(\mathsf{z}_{\kS+1,\nS}
+\mathsf{w}_{\kS,\nS})\bigr)\\
\tdred{\mathsf{Q}_1}(\boldsymbol{z}_{\nS},\boldsymbol{w}_{\nS})
=\tdviolet{q_{\nS}}\\
\begin{array}{l}
\text{for}\;\iS=1,\ldots,\mathsf{p}\\
\left\lfloor
\begin{array}{l}
\tdred{\mathsf{Q}_{1+\iS}}
(\boldsymbol{z}_{\nS},\boldsymbol{w}_{\nS})
=\frac{1}{2}(\tdred{\LS_{\iS}}\tdviolet{q_{\nS}}+z_{\iS+1,\nS}-
w_{\iS,\nS})\\
\end{array}
\right.\\[1cm]
\text{for}\;\kS=1,\ldots,\mathsf{p}\\
\left\lfloor
\begin{array}{l}
\tdred{\mathsf{Q}_{1+\mathsf{p}+\kS}}
(\boldsymbol{z}_{\nS},\boldsymbol{w}_{\nS})
=\frac{1}{2}(\tdred{\LS_{\kS}}\tdviolet{q_{\nS}}-z_{\kS+1,\nS}-
w_{\kS,\nS})\\
\tpyell{\prox_{\iota_{\{0\}}}}(2y_{\kS,\nS+1}-w_{\kS,\nS}) = 0.
\end{array}
\right.\\
\end{array}
\end{array}
\right.\\
\end{array}
\nonumber
\end{flalign*}
\pause
Then $({x}_{1,\nS})_{\nnn}$ converges weakly $\Pas$ to a 
$\ZS$-valued random variable. 
\end{exampleblock}
\end{frame}

%%%%%%%%%%%%%%%%%%%% EXPERIMENT %%%%%%%%%%%%%%%%%%%%%%%%%%%%%%%%%%
\begin{frame}{Numerical experiment}
\justifying
Consider the {\bf overlapping group lasso regression} described as
follows: 
\pause
$\tpblue{\HS}=\tpblue{\RR^{\mathsf{N}}}$ and, for every 
$\kS\in\{1,\dots,\mathsf{q}\}$,  
$\emp\neq\mathsf{I}_{\kS}\subset\{1,\dots,\mathsf{N}\}$ and 
\begin{equation}
\nonumber
\tdred{\LS_{\kS}}\colon\tpblue{\RR^{\mathsf{N}}}
\to\tpblue{\RR^{\card\,\mathsf{I}_{\kS}}}\colon
\mathsf{x}=(\upxi_{\jS})_{1\leq\jS\leq\mathsf{N}}\mapsto
(\upxi_{\jS})_{\jS\in\mathsf{I}_{\kS}}.
\end{equation}
Further, $\bigcup_{\kS=1}^{\mathsf{q}}\mathsf{I}_{\kS}
=\{1,\dots,\mathsf{N}\}$. The goal is to
\begin{equation}
\nonumber
\minimize{\mathsf{x}\in\RR^{\mathsf{N}}}{\frac{\upalpha}{2}
\|\mathsf{A}\mathsf{x}-\mathsf{b}\|^2+\dfrac{1}{\mathsf{q}}
\sum_{\kS=1}^{\mathsf{q}}\|\tdred{\LS_{\kS}}\mathsf{x}\|},
\end{equation}
where $\mathsf{A}\in\RR^{\mathsf{M}\times\mathsf{N}}$,
$\mathsf{b}\in\RR^\mathsf{M}$, and $\upalpha\in\RPP$.
\end{frame}

\begin{frame}{Numerical experiment}
\begin{itemize}
\justifying
\item
In the experiment $\mathsf{M}=1000$, $\mathsf{N}=3610$, 
$\mathsf{q}=40$, and
$\uplambda=5/\mathsf{q}^2$. 
\item\pause
The entries of 
$\mathsf{A}$ are i.i.d. samples from a $\mathcal{N}(0,1)$
distribution, and the entries of 
$\mathsf{b}$ are i.i.d. samples from a $\mathcal{N}(100,100)$
distribution. 
\item\pause
We split the function
$\|\mathsf{A}\mathsf{x}-\mathsf{b}\|^2$ into $30$ blocks of $40$ 
entries each, where the entries are selected in order without 
overlap.
\item\pause
Finally, 
\begin{align}
\nonumber
(\forall\kS\in\{1,\dots,\mathsf{p}\})\;\;
\mathsf{I}_{\kS}=\{90\kS-89,\ldots,90\kS+10\}.
\end{align}
\end{itemize}
\end{frame}

\begin{frame}{Numerical experiment}
\begin{figure}[ht!]
\begin{tikzpicture}[scale=0.545]
\definecolor{darkgray176}{RGB}{176,176,176}
\begin{axis}[height=8cm,width=1.818\columnwidth, legend columns=1 
cell align={left}, xmin=0, xmax=1200, ymin=-100, ymax=0,
xtick distance=100,
ytick distance=25,
x grid style={darkgray176},xmajorgrids,
y grid style={darkgray176},ymajorgrids,
tick label style={font=\Large}, 
legend cell align={left},
legend style={at={(0.02,0.15)},anchor=west},
axis line style=thick,]
\addplot [ultra thick, dgreen]
table {figures/ex2/F1_sol.txt};
\addplot [ultra thick, orng]
table {figures/ex2/F2_sol.txt};
\addplot [ultra thick, dblue]
table {figures/ex2/F3_sol.txt};
%\addplot [ultra thick, dashed, dviolet]
%table {figures/ex2/CH_sol.txt};
%\addplot [ultra thick, dashed, dred]
%table {figures/ex2/RN_sol.txt};
\end{axis}
\end{tikzpicture}
\caption{
\justifying
Normalized error $20\log(\|x_{1,\nS}-x_\infty\|/
\|x_{1,0}-x_\infty\|)$ (dB) versus execution time (s).
{\bf\color{dgreen} Green}: Framework~1.
{\bf\color{orng} Orange}: Framework~2. 
{\bf\color{dblue} Blue}: Framework~3.
%{\color{dred} Dashed red}: Algorithm~\eqref{e:71}.
%{\color{dviolet} Dashed violet}: Algorithm~\eqref{e:72}.}
}
\label{fig:ex1}
\end{figure}
\end{frame}


%%%%%%%%%%%%%%%%%%%% DISCUSSION %%%%%%%%%%%%%%%%%%%%%%%%%%%%%%%%%%
\begin{frame}{Discussion}
\justifying
The frameworks differ in terms of storage
requirements, use of \tdgreen{proximity} operators, and use of 
\tdred{linear} operators.
\begin{itemize}
\justifying
\item\pause
Framework 1: It stores $2\mathsf{p}+3$ vectors and, for each of 
the $\mathsf{p}+1$ random activation indices, there is one
\tdgreen{proximity} evaluation.
\item\pause
Framework 2: It stores $4\mathsf{p}+5$ vectors. The first 
$\mathsf{p}+1$ indices involve a \tdgreen{proximity} operator.
The \tdred{linear} operators are used only if index 
$\mathsf{p}+2$ is activated.
\item\pause
Framework 3: It stores $4\mathsf{p}+3$ vectors. 
Moreover, out of the $2\mathsf{p}+1$ random activation 
indices, those in $\{1,\ldots,\mathsf{p}+1\}$ involve a
\tdgreen{proximity} operator, while those in 
$\{\mathsf{p}+2,\ldots,\mathsf{p}+\mathsf{r}+1\}$
do not require it. 
\end{itemize}
\end{frame}

\begin{frame}{Discussion}
\justifying
Although Framework~1 is the most efficient in terms of storage, it
may not always be the fastest, especially when \tdgreen{proximity} 
operators are computationally expensive or when the \tdred{linear} 
operators are costly, which is the case in the experiment.
\end{frame}

\begin{frame}
\titlepage
\centering{
\includegraphics[height=0.6cm]{ncsu.eps}}
\end{frame}
\end{document}
